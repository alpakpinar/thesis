% ABSTRACT

There are multiple sources of astrophysical evidence which support the presence of dark matter (DM), which stands out as
one of the open questions in the standard model (SM) of particle physics. One avenue to look for DM production is at 
the Large Hadron Collider (LHC), where the production of DM can be detected as events with large missing transverse momentum ($\ptmiss$).
This thesis documents a search for new DM particles using proton-proton collisions at the LHC, recorded with the Compact Muon Solenoid (CMS)
detector, at a center of mass energy of 13 TeV. In this search, the target signature is a Higgs boson, produced via the Vector Boson Fusion (VBF) process, 
decaying into a pair of DM particles, resulting in two energetic jets and large $\ptmiss$ in the final state. To estimate the background processes,
multiple control regions are defined and a simultaneous fit to data over all regions is performed. The data for this search was collected in 2017 and 2018, 
during Run 2 of the LHC. A full result corresponding to an integrated luminosity of $137 \ \textrm{fb}^{-1}$
is also obtained by statistically combining this analysis result with the already published 2016 analysis. No excess of events is observed over
expected SM backgrounds. The results are interpreted in the context of Higgs-portal models, where upper bounds are set on 
the branching ratio for the SM Higgs boson decaying to invisible DM particles.


