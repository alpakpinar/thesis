% This file contains all the necessary setup and commands to create
% the preliminary pages according to the buthesis.sty option.

\title{Search for invisible decays of the Higgs boson produced via
vector boson fusion at LHC with the CMS detector Run-II data}

\author{Alp Akpinar}

% Type of document prepared for this degree:
%   1 = Master of Science thesis,
%   2 = Doctor of Philosophy dissertation.
\degree=2

\prevdegrees{B.Sc., Bogazici University, 2018}

\department{Department of Physics}

% Degree year is the year the diploma is expected, and defense year is
% the year the dissertation is written up and defended. Often, these
% will be the same, except for January graduation, when your defense
% will be in the fall of year X, and your graduation will be in
% January of year X+1
\defenseyear{2023}
\degreeyear{2023}

% For each reader, specify appropriate label {First, Second, Third},
% then name, and title. IMPORTANT: The title should be:
%   "Professor of Electrical and Computer Engineering",
% or similar, but it MUST NOT be:
%   Professor, Department of Electrical and Computer Engineering"
% or you will be asked to reprint and get new signatures.
% Warning: If you have more than five readers you are out of luck,
% because it will overflow to a new page. You may try to put part of
% the title in with the name.
\reader{First}{Zeynep Demiragli, PhD}{Assistant Professor of Physics}
\reader{Second}{James Rohlf, PhD}{Professor of Physics}

% The Major Professor is the same as the first reader, but must be
% specified again for the abstract page. Up to 4 Major Professors
% (advisors) can be defined. 
\numadvisors=1
\majorprof{Zeynep Demiragli, PhD}{Assistant Professor of Physics}
% \majorprofb{First M. Last, PhD}{{Professor of Astronomy}}
%\majorprofc{First M. Last, PhD}{{Professor of Astronomy}}
%\majorprofd{First M. Last, PhD}{{Professor of Biomedical Engineering}}

%%%%%%%%%%%%%%%%%%%%%%%%%%%%%%%%%%%%%%%%%%%%%%%%%%%%%%%%%%%%%%%%  

%                       PRELIMINARY PAGES
% According to the BU guide the preliminary pages consist of:
% title, copyright (optional), approval,  acknowledgments (opt.),
% abstract, preface (opt.), Table of contents, List of tables (if
% any), List of illustrations (if any). The \tableofcontents,
% \listoffigures, and \listoftables commands can be used in the
% appropriate places. For other things like preface, do it manually
% with something like \newpage\section*{Preface}.

% This is an additional page to print a boxed-in title, author name and
% degree statement so that they are visible through the opening in BU
% covers used for reports. This makes a nicely bound copy. Uncomment only
% if you are printing a hardcopy for such covers. Leave commented out
% when producing PDF for library submission.
%\buecethesistitleboxpage

% Make the titlepage based on the above information.  If you need
% something special and can't use the standard form, you can specify
% the exact text of the titlepage yourself.  Put it in a titlepage
% environment and leave blank lines where you want vertical space.
% The spaces will be adjusted to fill the entire page.
\maketitle
\cleardoublepage

% The copyright page is blank except for the notice at the bottom. You
% must provide your name in capitals.
\copyrightpage
\cleardoublepage

% Now include the approval page based on the readers information
% Once the approval page is approved by the Mugar Library staff, please
% comment out the "\approvalpagewithcomment" line and uncomment "\approvalpage"
\approvalpagewithcomment
%\approvalpage
\cleardoublepage

% Here goes your favorite quote. This page is optional.
\newpage
%\thispagestyle{empty}
\phantom{.}
\vspace{4in}

\begin{singlespace}
\begin{quote}
  \textit{All things are difficult before they are easy.}\\
  \textit{- Dr. Thomas Fuller}\\*
\end{quote}
\end{singlespace}

% \vspace{0.7in}
%
% \noindent
% [The descent to Avernus is easy; the gate of Pluto stands open night
% and day; but to retrace one's steps and return to the upper air, that
% is the toil, that the difficulty.]

\cleardoublepage

% The acknowledgment page should go here. Use something like
% \newpage\section*{Acknowledgments} followed by your text.
\newpage
\section*{\centerline{Acknowledgments}}
I am very grateful to my academic advisor, Prof. Zeynep Demiragli, for all her guidance and support since the start of my PhD. 
Under her influence, I had the chance to learn a lot and work with a lot of great people in very interesting projects. 
I'd also like to thank my committee, Prof. James Rohlf, Prof. David Sperka, Prof. Martin
Schmaltz and Prof. Shyamsunder Erramilli. Also thanks to Prof. Emanuel Katz for attending my preliminary oral exam and departmental seminar. 

I'd like to thank all the faculty, researchers and engineers I worked with in the CMS group at Boston University, 
from whom I learned a lot. Special thanks to Daniel Gastler, for teaching me a lot about C++ programming and good software 
development practices. Also special thanks to Andreas Albert, for
teaching me a lot about physics, communication and other technical skills, and inspiring me in my journey throughout my PhD. I'd also like to thank
Prof. James Rohlf for his very detailed comments on this thesis. In addition, I'd like to thank my student colleagues and friends at the high energy
physics group here at Boston University.

I'd like to thank my family for their support all the way from Turkey, even under the most stressful of situations. I'd also like to thank 
my girlfriend Natalia Valles, for her emotional support throughout my program, and making Boston a beautiful and more fun city to live
for me. 

\vskip 1in

\noindent
\cleardoublepage

% The abstractpage environment sets up everything on the page except
% the text itself.  The title and other header material are put at the
% top of the page, and the supervisors are listed at the bottom.  A
% new page is begun both before and after.  Of course, an abstract may
% be more than one page itself.  If you need more control over the
% format of the page, you can use the abstract environment, which puts
% the word "Abstract" at the beginning and single spaces its text.

\begin{abstractpage}
% ABSTRACT

There are multiple sources of astrophysical evidence which support the presence of dark matter (DM), which stands out as
one of the open questions in the standard model (SM) of particle physics. One avenue to look for DM production is at 
the Large Hadron Collider (LHC), where the production of DM can be detected as events with large missing transverse momentum ($\ptmiss$).
This thesis documents a search for new DM particles using proton-proton collisions at the LHC, recorded with the Compact Muon Solenoid (CMS)
detector, at a center of mass energy of 13 TeV. In this search, the target signature is a Higgs boson, produced via the Vector Boson Fusion (VBF) process, 
decaying into a pair of DM particles, resulting in two energetic jets and large $\ptmiss$ in the final state. To estimate the background processes,
multiple control regions are defined and a simultaneous fit to data over all regions is performed. The data for this search was collected in 2017 and 2018, 
during Run 2 of the LHC. A full result corresponding to an integrated luminosity of $137 \ \textrm{fb}^{-1}$
is also obtained by statistically combining this analysis result with the already published 2016 analysis. No excess of events is observed over
expected SM backgrounds. The results are interpreted in the context of Higgs-portal models, where upper bounds are set on 
the branching ratio for the SM Higgs boson decaying to invisible DM particles.



\end{abstractpage}
\cleardoublepage

% Now you can include a preface. Again, use something like
% \newpage\section*{Preface} followed by your text

% Table of contents comes after preface
\tableofcontents
\cleardoublepage

% If you do not have tables, comment out the following lines
\newpage
\listoftables
\cleardoublepage

% If you have figures, uncomment the following line
\newpage
\listoffigures
\cleardoublepage

% List of Abbrevs is NOT optional (Martha Wellman likes all abbrevs listed)
% \chapter*{List of Abbreviations}

% {\bf The list below must be in alphabetical order as per BU library instructions or it will be returned to you for re-ordering.}

% \begin{center}
%   \begin{tabular}{lll}
%     \hspace*{2em} & \hspace*{1in} & \hspace*{4.5in} \\
%     ALICE & \dotfill & A Large Ion Collider Experiment \\
%     ATLAS & \dotfill & A Toroidal LHC Apparatus \\
%     BR   & \dotfill & Branching ratio \\
%     BSM  & \dotfill & Beyond the Standard Model of Particle Physics \\
%     CERN & \dotfill & European Council for Nuclear Research \\
%     CMS  & \dotfill & Compact Muon Solenoid \\
%     CR   & \dotfill & Control region \\
%     DM   & \dotfill & Dark matter \\
%     ECAL & \dotfill & Electromagnetic calorimeter \\
%     HCAL & \dotfill & Hadronic calorimeter \\
%     HF   & \dotfill & Forward hadronic calorimeter \\
%     JME  & \dotfill & JetMET physics object group \\
%     LHC  & \dotfill & Large Hadron Collider \\
%     LHCb & \dotfill & Large Hadron Collider beauty experiment \\
%     LO   & \dotfill & Leading order \\
%     MC   & \dotfill & Monte Carlo simulation \\
%     MET  & \dotfill & Missing transverse energy \\
%     NLO  & \dotfill & Next-to-leading order \\
%     POG  & \dotfill & Physics object group \\
%     PF   & \dotfill & Particle flow algorithm \\
%     PU   & \dotfill & Pileup \\
%     SF   & \dotfill & Scale factor \\
%     SM   & \dotfill & Standard Model of Particle Physics \\
%     SR   & \dotfill & Signal region \\
%     VBF  & \dotfill & Vector boson fusion \\
%   \end{tabular}
% \end{center}
\cleardoublepage

% END OF THE PRELIMINARY PAGES

\newpage
\endofprelim
